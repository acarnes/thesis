\RequirePackage{luatex85}
\documentclass{standalone}
\usepackage[compat=1.1.0]{tikz-feynman}

\begin{document}

% phi4 vertex
%\feynmandiagram [baseline=(current bounding box.center), small, horizontal=i1 to f2] {
%  {i1, i2} -- c [dot, label=0:\(z\)] -- {f1, f2},
%};
% tadpole
\feynmandiagram [baseline=(current bounding box.center), horizontal=a to c] {
  a [dot, label=90:\(x_1\)] -- b [dot, label=-90:\(z\) ] -- [out=45, in=135, loop, min distance=2cm] b,
  b -- c [dot, label=90:\(x_2\)],
};
% propagator
\feynmandiagram [baseline=(current bounding box.center), horizontal= a to b] {
  a [dot, label=90:\(x_1\)] -- b [dot, label=90:\(x_2\)],
};
% figure eight
\feynmandiagram [baseline=(current bounding box.center), horizontal= b to b] {
  b [dot, label=0:\(z\) ] -- [out=45, in=135, loop, min distance=2cm] b -- [out=-45, in=-135, loop, min distance=2cm] b,
};

%\begin{tikzpicture}
%  \begin{feynman}
%    \diagram [horizontal=a to b] {
%      a -- [fermion] b,
%    };
%    \draw (b) arc [start angle=180, end angle=-180, radius=0.7cm];
%  \end{feynman}
%\end{tikzpicture}

%\feynmandiagram [horizontal=a to b] {
%  i1 -- [fermion] a -- [fermion] i2,
%  a -- [photon] b,
%  f1 -- [fermion] b -- [fermion] f2,
%};

%\feynmandiagram [horizontal=a to b] {
%a [particle=\(\mu^{-}\)] -- [fermion] b -- [fermion] f1 [particle=\(\nu_{\mu}\)],
%b -- [boson, edge label=\(W^{-}\)] c,
%f2 [particle=\(\overline \nu_{e}\)] -- [fermion] c -- [fermion] f3 [particle=\(e^{-}\)],
%};

%\begin{tikzpicture}
%\begin{feynman}
%\vertex (a) {\(\mu^{-}\)};
%\vertex [right=of a] (b);
%\vertex [above right=of b] (f1) {\(\nu_{\mu}\)};
%\vertex [below right=of b] (c);
%\vertex [above right=of c] (f2) {\(\overline \nu_{e}\)};
%\vertex [below right=of c] (f3) {\(e^{-}\)};
%\diagram* {
%(a) -- [fermion] (b) -- [fermion] (f1),
%(b) -- [boson, edge label'=\(W^{-}\)] (c),
%(c) -- [anti fermion] (f2),
%(c) -- [fermion] (f3),
%};
%\end{feynman}
%\end{tikzpicture}

\end{document}
