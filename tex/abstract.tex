% Write in only the text of your abstract, all the extra heading jargon is automatically taken care of
\begin{abstract}
In 2012 two collaborations at the Large Hadron Collider announced the discovery of a new particle with properties similar to the Standard Model Higgs Boson. In order to determine whether the boson discovered with a mass of 125 GeV is actually the Standard Model Higgs boson, all of the different ways the particle can decay need to be investigated. If the probabilities for the different decays do not match the predictions of the Standard Model then this would imply new physics. 

This dissertation presents the search for the Standard Model Higgs boson decaying to $\mu^{+}\mu^{-}$. The search uses the $35.9\pm0.9$~fb$^{-1}$ of $\sqrt{s}$ = 13 TeV proton-proton collision data recorded by the CMS detector in 2016. The signal strength ($\mu = (\sigma\mathcal{B})/(\sigma\mathcal{B})_{SM}$) is measured at $0.7^{+1.1}_{-1.0}$ for $m_H=125$ GeV, where $\sigma$ is the Higgs production cross section and $\mathcal{B}$ is the branching fraction to muons. The observed and expected upper limits on the signal strength at a 95 \% confidence level are presented for Higgs masses in the range 120 to 130 GeV. The observed and expected upper limits on the signal strength at a mass of 125 GeV are 2.64 and 2.08 respectively. The significance is reported in the same range, and the observed and expected significance at $m_H=125$ GeV are 0.74$\sigma$ and 0.98$\sigma$ respectively. 

Combined results for 5.0 $fb^{-1}$ of 7 TeV, 19.8 $fb^{-1}$ of 8 TeV, and 35.9 $fb^{-1}$ of 13 TeV data are also presented. For $m_H=125$ GeV, the combination yields a measured signal strength of $0.9^{+1.0}_{-0.9}$, observed (expected) upper limits at 95\% confidence of 2.64 (1.89), and an observed (expected) significance of 0.98 (1.09)$\sigma$. The results correspond to an upper limit on the $H\rightarrow\mu^+\mu^-$ branching fraction of 5.7x10$^{-4}$. These results provide the best results to date on the Higgs coupling to second generation fermions. No deviations from the Standard Model are observed. 
\end{abstract}
