% Write in only the text of your abstract, all the extra heading jargon is automatically taken care of
\begin{abstract}
In 2012 two collaborations at the Large Hadron Collider announced the discovery of a new particle with properties similar to the Standard Model Higgs Boson. In order to determine whether the boson discovered with a mass of 125 GeV is actually the Standard Model Higgs, all of the different ways the particle can decay need to be investigated. If the probabilities for the different decays do not match the predictions of the Standard Model then this would imply new physics. 

This dissertation presents the search for the Standard Model Higgs Boson decaying to $\mu^{+}\mu^{-}$. The search uses the $35.9\pm0.9$~fb$^{-1}$ of $\sqrt{s}$ = 13 TeV proton-proton collision data recorded by the CMS detector in 2016. The observed and expected upper limits on the rate at a 95 \% confidence level are presented for Higgs masses in the range 120 to 130 GeV. The expected and observed upper limits at a mass of 125 GeV are x.xx and $1.98^{+0.81}_{-0.57}$ $\times$ SM respectively. These results provide the best results to date on the Higgs coupling to second generation fermions. No deviations from the Standard Model are observed. 
\end{abstract}
