\chapter{RESULTS} \label{results}

\section{Fusce Eget Tempus Lectus, }

%\hline
\begin{algorithm}  {Euclid’s algorithm}
%\hline %
\singlespacing

\begin{algorithmic}[1]
%\caption{Euclid’s algorithm}\label{alg:euclid}
\Procedure{Euclid}{$a,b$}\Comment{The g.c.d. of a and b}
\State $r\gets a\bmod b$
\While{$r\not=0$}\Comment{We have the answer if r is 0}
\State $a\gets b$
\State $b\gets r$
\State $r\gets a\bmod b$
\EndWhile\label{euclidendwhile}
\State \textbf{return} $b$\Comment{The gcd is b}
\EndProcedure
%\hline
\end{algorithmic}
\end{algorithm}









\proposition{The Upsilon Function}

(1) If $\beta>0$ and $\alpha\neq0$, then for all $n\geq-1$,

$$I_{n}(c;\alpha; \beta; \delta) = - \frac{e^{\alpha c}}{\alpha} \sum_{i=0}^{n}(\frac{\beta}{\alpha})^{n-i} Hh_{i}(\beta c -\delta)$$

$$+ (\frac{\beta}{\alpha})^{n+1} \frac{\sqrt{2 \pi}}{\beta} e^{\frac{\alpha \delta}{\beta}+\frac{\alpha^{2}}{2\beta^{2}}} \phi(-\beta c + \delta + \frac{\alpha}{\beta})$$
(2) If $\beta<0$ and $\alpha<0$, then for all $x \geq -1$

$$I_{n}(c;\alpha; \beta; \delta) = - \frac{e^{\alpha c}}{\alpha} \sum_{i=0}^{n}(\frac{\beta}{\alpha})^{n-i} Hh_{i}(\beta c -\delta)$$

$$- (\frac{\beta}{\alpha})^{n+1} \frac{\sqrt{2 \pi}}{\beta} e^{\frac{\alpha \delta}{\beta}+\frac{\alpha^{2}}{2\beta^{2}}} \phi(\beta c - \delta - \frac{\alpha}{\beta})$$

\begin{proof}{Case 1.}

$\beta>0$ and $\alpha\neq0$. Since, for any constant $\alpha$ and $n \geq 0$, $e^{\alpha x} Hh_{n}(\beta x - \delta) \rightarrow 0$ as $x \rightarrow \infty$ thanks to (B4), integration by parts leads to

$$I_{n}=-\frac{1}{\alpha}Hh(\beta c -\delta) e^{\alpha c} + \frac{\beta}{\alpha}\int_{c}^{\infty} e^{\alpha x} Hh_{n-1}(\beta c - \delta)dx$$

In other words, we have a recursion, for $n \geq 0$, $I_{n}=-(e^{\alpha c}{\alpha})Hh_{n}(\beta c - \delta) + (\frac{\beta}{\alpha})I_{n-1}$ with

$$I_{-1}=\sqrt{2 \pi} \int_{c}{\infty}e^{\alpha x}\varphi(-\beta x +\delta)dx$$

$$=\frac{\sqrt{2 \pi}}{\beta} e^{\frac{\alpha \delta}{\beta}+\frac{\alpha^{2}}{2 \beta^{2}}}\phi(-\beta c + \delta +\frac{\alpha}{\beta})$$

Solving it yields, for $n \geq -1$,

$$I_{n}=-\frac{e^{\alpha c}}{\alpha}\sum_{i=0}^{n}(\frac{\beta}{\alpha})^{i}Hh_{n-i}(\beta c+\delta) + (\frac{\beta}{\alpha})^{n+1}I_{-1}$$

$$=-\frac{e^{\alpha c}}{\alpha}\sum_{i=0}^{n}(\frac{\beta}{\alpha})^{n-i} Hh_{i}(\beta c+\delta)$$

$$+ (\frac{\beta}{\alpha})^{n+1}\frac{\sqrt{2 \pi}}{\beta} e^{\frac{\alpha \delta}{\beta}+\frac{\alpha^{2}}{2 \beta^{2}}}\phi(-\beta c + \delta +\frac{\alpha}{\beta})$$

where the sum over an empty set is defined to be zero.
\end{proof}

\begin{proof}{Case2.} $\beta<0$ and $\alpha<0$. In this case, we must also have, for $n \geq 0$ and any constant $\alpha<0, e^{\alpha x}Hh_{n}(\beta x -\delta) \rightarrow 0$ as

$x \rightarrow \infty$, thanks to (B5). Using integration by parts, we again have the same recursion, for $n \geq 0, I_{n}=-(e^{\alpha c}/\alpha)Hh_{n}(\beta c - \delta)+(\beta / \alpha)I_{n-1}$, but with a different initial condition

$$I_{-1}=\sqrt{2 \pi}\int_{c}^{\infty}e^{\alpha x}\varphi(-\beta x + \delta)dx$$

$$=-\frac{\sqrt{2 \pi}}{\beta} exp\{\frac{\alpha \delta}{\beta}+\frac{\alpha^{2}}{2 \beta^{2}}\}\phi(\beta c - \delta -\frac{\alpha}{\beta})$$

Solving it yields (B8), for $n \geq -1$.

\end{proof}

Finally, we sum the double exponential and the normal random variables

Proposition B.3.

Suppose $\{\xi_{1},\xi_{2},...\}$ is a sequence of i.i.d. exponential random variables with rate $\eta>0$, and Z is a normal variable with distribution $N(0,\sigma^{2})$. Then for every $ n \geq 1$, we have: (1) The density functions are given by:

$$f_{Z+\sum_{i=1}^{n}\xi_{i}}(t)=(\sigma\eta)^{n}\frac{e^{(\sigma\eta)^{2}/2}}{\sigma\sqrt{2\pi}}e^{-t\eta}Hh_{n-1}(-\frac{t}{\sigma}+\sigma\eta)$$

$$f_{Z-\sum_{i=1}^{n}\xi_{i}}(t)=(\sigma\eta)^{n}\frac{e^{(\sigma\eta)^{2}/2}}{\sigma\sqrt{2\pi}}e^{-t\eta}Hh_{n-1}(\frac{t}{\sigma}+\sigma\eta)$$
(2) The tail probabilities are given by

$$P(Z+\sum_{i=1}^{n}\xi_{i}\geq x) = (\sigma\eta)^{n}\frac{e^{(\sigma\eta)^{2}/2}}{\sigma\sqrt{2\pi}}e^{-t\eta}I_{n-1}(x;-\eta,-\frac{1}{\sigma},-\sigma\eta)$$

$$P(Z-\sum_{i=1}^{n}\xi_{i}\geq x) = (\sigma\eta)^{n}\frac{e^{(\sigma\eta)^{2}/2}}{\sigma\sqrt{2\pi}}e^{-t\eta}I_{n-1}(x;\eta,\frac{1}{\sigma},-\sigma\eta)$$

Proof. Case 1. The densities of $Z+\sum_{i=1}^{n}\xi_{i}$, and $Z-\sum_{i=1}^{n}\xi_{i}$. We have

$$f_{Z+\sum_{i=1}^{n}\xi_{i}}(t)=\int_{-\infty}^{\infty}f_{\sum_{i=1}^{n}\xi_{i}}(t-x)f_{Z}(x)dx$$

$$=e^{-t\eta}(\eta^{n})\int_{-\infty}{t}\frac{e^{x\eta}(t-x)^{n-1}}{(n-1)!}\frac{1}{\sigma\sqrt{2\pi}}e^{-x^{2}/(2\sigma^{2})}dx$$

$$=e^{-t\eta}(\eta^{n})e^{(\sigma\eta)^{2}/(2)}\int_{-\infty}{t}\frac{(t-x)^{n-1}}{(n-1)!}\frac{1}{\sigma\sqrt{2\pi}}e^{-(x-\sigma^{2}\eta)^{2}/(2\sigma^{2})}dx$$

Letting $y=(x-\sigma^{2}\eta)/\sigma$ yields

$$f_{Z+\sum_{i=1}^{n}\xi_{i}}(t)=e^{-t\eta}(\eta^{n})e^{(\sigma\eta)^{2}/(2)}\sigma^{n-1}$$

$$\times\int_{-\infty}^{t/\sigma-\sigma\eta}\frac{(t/\sigma - y -\sigma\eta)^{n-1}}{(n-1)!}\frac{1}{\sqrt{2\pi}}e^{-y^{2}/2}dy$$

$$=\frac{e^{(\sigma\eta)^{2}/2}}{\sqrt{2\pi}}(\sigma^{n-1}\eta^{n})e^{-t\eta}Hh_{n-1}(-t/\sigma + \sigma\eta)$$

because $(1/(n-1)!)\int_{-\infty}{a}(a-y)^{n-1}e^{-y^{2}/2}dy=Hh_{n-1}(a)$. The derivation of $f_{Z+\sum_{i=1}^{n}\xi_{i}}(t)$ is similar.

Case 2. $P(Z+\sum_{i=1}^{n}\xi_{i}\geq x)$ and $P(Z-\sum_{i=1}^{n}\xi_{i}\geq x)$. From (B9), it is clear that

$$P(Z+\sum_{i=1}^{n}\xi_{i}\geq x)=\frac{(\sigma\eta)^{n}e^{(\sigma\eta)^{2}/2}}{\sigma\sqrt{2\pi}}\int_{x}^{\infty}e^{(-i\eta)}Hh_{n-1}(-\frac{t}{\sigma}+\sigma\eta)dt$$

$$=\frac{(\sigma\eta)^{n}e^{(\sigma\eta)^{2}/2}}{\sigma\sqrt{2\pi}}I_{n-1}(x;-\eta,-\frac{1}{\sigma},-\sigma\eta)dt$$

by (B6). We can compute
$P(Z-\sum_{i=1}^{n}\xi_{i}\geq x)$ similarly.

\theorem{Theorem} With $\pi_{n}:= P(N(t)=n)=e^{-\lambda T}(\lambda T)^{n}/n!$ and $I_{n}$ in Proposition \ref{first}.
, we have

$$P(Z(T)\geq a)=\frac{e^{(\sigma \eta_{1})^{2} T/2}}{\sigma \sqrt{2 \pi T}} \sum_{n=1}^{\infty} \pi_{n} \sum_{k=1}^{n} P_{n,k}(\sigma\sqrt{T}\eta_{1})^{k}\times I_{k-1}(a-\mu T; -\eta_{1},-\frac{1}{\sigma\sqrt{T}},-\sigma\eta_{1}\sqrt{T})$$

$$+\frac{e^{(\sigma\eta_{2})^{2}T/2}}{\sigma\sqrt{2\pi T}}\sum_{n=1}^{\infty}\pi_{n}\sum_{k=1}^{n}Q_{n,k}(\sigma\sqrt{T}\eta_{2})^{k}$$

$$\times I_{k-1}(a-\mu T; \eta_{2},\frac{1}{\sigma\sqrt{T}},-\sigma\eta_{2}\sqrt{T})$$

$$+\pi_{0}\phi(-\frac{a-\mu T}{\sigma\sqrt{T}})$$

Proof by the decomposition (B2)

$$P(Z(T) \geq a)= \sum_{n=0}^{\infty}\pi_{n} P(\mu T +\sigma\sqrt{T} Z + \sum_{j=1}^{n}Y_{j} \geq a)$$

$$=\pi_{0}P(\mu T +\sigma\sqrt{T} Z  \geq a)$$

$$+\sum_{n=1}^{\infty}\pi_{n}\sum_{k=1}^{n}P_{n,k} P(\mu T +\sigma\sqrt{T} Z + \sum_{j=1}^{n}\xi_{j}^{+} \geq a)$$

$$+\sum_{n=1}^{\infty}\pi_{n}\sum_{k=1}^{n}Q_{n,k} P(\mu T +\sigma\sqrt{T} Z - \sum_{j=1}^{n}\xi_{j}^{-} \geq a)$$

The result now follows via (B11) and (B12) for $\eta_{1} > 1$ and $\eta_{2} >0$. 