\chapter{CONCLUSIONS} \label{conclusions}

A Higgs boson with a mass of 125 GeV was discovered in 2012 \cite{atlasdiscovery,cmsdiscovery2012,cmsdiscovery2013}, but the exact behavior of the Higgs coupling to fermions remains undetermined. The SM predicts a 2.2x$10^{-4}$ branching fraction for $H\rightarrow\mu^+\mu^-$, but various BSM theories predict branching fractions greater than the SM value \cite{gtsm1, gtsm2, gtsm3}. Moreover, the SM also predicts that the branching fractions for the different fermions are proportional to the square of the fermion mass, and specifically that $\mathcal{B}_{\tau\tau}/\mathcal{B}_{\mu\mu} = m^2_\tau/m^2_\mu$. Measuring the decay rates to the different fermions with greater precision reduces the number of acceptable theories, and if any new measurements happen to rule out the Standard Model, then this implies new physics.  

ATLAS and CMS used 2011 and 2012 LHC collisions to investigate the branching fractions of a 125 GeV Higgs boson to fermions and to muons in particular. The CMS experiment reported an observed (expected) upper limit on the $H\rightarrow\mu^+\mu^-$ signal strength of 7.4 (6.5) at 95\% confidence \cite{cmshmumu2012}. The results used 5.0 $fb^{-1}$ of 7 TeV data and 19.7 $fb^{-1}$ of 8 TeV data. At the same time, ATLAS reported observed (expected) upper limits of 7.1 (7.2) at 95\% confidence using 4.5 $fb^{-1}$ of 7 TeV data and 20 $fb^{-1}$ of 8 TeV data \cite{atlashmumu2012}. 

This dissertation presents new results for $H\rightarrow\mu^+\mu^-$, using 35.9 $fb^{-1}$ of 13 TeV data collected by the CMS experiment in 2016. A new observed (expected) upper limit on the signal strength is reported at 2.64 (2.08) at 95\% confidence. An observed (expected) significance of 0.75$\sigma$ (0.98$\sigma$)  and a best fit signal strength of $0.7^{+1.1}_{-1.0}$ are also reported. Meanwhile, the latest results from ATLAS report an observed (expected) upper limit of 3.0 (3.1) using 36.1 $fb^{-1}$ of 13 TeV data \cite{atlashmumu2017}. 

Combined results for 5.0 $fb^{-1}$ of 7 TeV, 19.8 $fb^{-1}$ of 8 TeV, and 35.9 $fb^{-1}$ of 13 TeV CMS data are also presented. For $m_H=125$ GeV, the combination yields a measured signal strength of $0.9^{1.0}_{-0.9}$, observed (expected) upper limits at 95\% confidence of 2.64 (1.89), and an observed (expected) significance of 0.98 (1.09)$\sigma$. The results correspond to an upper limit on the $H\rightarrow\mu^+\mu^-$ branching fraction of 5.7x10$^{-4}$. Any theories predicting a larger rate are ruled out at this confidence level. Upon combining 4.5 $fb^{-1}$ of 7 TeV, 20 $fb^{-1}$ of 8 TeV, and 36.1 $fb^{-1}$ of 13 TeV data, ATLAS reports an observed (expected) upper limit of 2.8 (2.9) at 95\% confidence \cite{atlashmumu2017}.

Scaling the signal and background PDFs from the 13 TeV analysis of this dissertation to different integrated luminosities provides estimates of the expected p-value versus the integrated luminosity. About 400 $fb^{-1}$ of data is required to observe $H\rightarrow\mu^+\mu^-$ with a p-value of 3$\sigma$. The entire extrapolation can be seen in Figure \ref{fig:pvalprojection}.

To date, observations of the Higgs branching fractions to fermions show no significant deviations from the Standard Model predictions \cite{cmshiggstau2017,cmshiggsbb2017, cmshmumu2017, atlashmumu2017}. In particular, the latest CMS results on $H\rightarrow b\bar{b}$ \cite{cmshiggsbb2017} and $H\rightarrow \tau^+\tau^-$ \cite{cmshiggstau2017} report signal strengths of $1.06^{+0.31}_{-0.29}$ and $0.98^{+0.18}_{-0.18}$ for combined 2011, 2012, and 2016 data at $m_H=125$ GeV. The latest measurement of the signal strength for $H\rightarrow\mu^+\mu^-$ ($0.9^{1.0}_{-0.9}$) is also consistent with the SM for $m_H=125$ GeV. 

