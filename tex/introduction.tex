\chapter{INTRODUCTION} \label{intro}

The Standard Model of particle physics is an extremely successful theory shown to correctly predict the behavior of the particles and forces which make up the most basic constituents of the universe. In fact, it correctly describes all of the forces known except for gravity. All of the elementary particles detailed in the Standard Model have been discovered except for one, the Higgs boson. As explained by special relativity, potential energy can be realized as mass and it is the potential from the interaction with the nonzero Higgs field that bestows mass onto the massive elementary particles in the Standard Model. On July 4, 2012 two collaborations at the Large Hadron Collider (LHC), the A Toroidal LHC Apparatus (ATLAS) and Compact Muon Solenoid (CMS), announced the discovery of a new boson at 125 GeV with properties similar to the Standard Model Higgs \cite{atlasdiscovery,cmsdiscovery2012,cmsdiscovery2013}. This discovery was fueled by the investigation into the Higgs decays to the vector bosons ZZ and ${\rm \gamma\gamma}$. Soon after, evidence for the Higgs coupling to matter was found through the ${\rm \tau^{+}\tau{-}}$ and ${\rm b\bar{b}}$ decays \cite{cmshiggstau,cmshiggsbb,cmshiggsferm,atlashiggsbb}. Whether the newly discovered boson is indeed the expected Standard Model Higgs remains to be determined. Insofar, all of the different decay modes will be investigated to search for deviations from the Standard Model predictions.

This leads to the study of the Higgs decay to $\mu^{+}\mu^{-}$. Although this decay is the smallest branching fraction expected to be detected \cite{smallestbranch1,smallestbranch2}, the dimuon decay offers high efficiency and excellent momentum resolution, which should lead to a narrow peak over the falling background, mostly Drell Yan events. The tiny branching fraction enables greater sensitivity to small deviations from the predicted decay rate and in this respect offers an advantage over other channels where a miniscule deviation could be drowned out. Furthermore, the Higgs coupling to second generation fermions remains to be determined. 

This dissertation presents the search for the Standard Model Higgs Boson decaying to $\mu^{+}\mu^{-}$ using the proton-proton collision data recorded by the CMS experiment in 2016. In order to maximize the data available for the search, the first machine learning in the L1 Trigger system at the LHC was developed and deployed for 2016 data collection. To further maximize the sensitivity of the search, an additional machine learning technique is invented to categorize events based upon the detector resolution and the event kinematics. The search looks for a Higgs boson with a mass between 120 and 130 GeV and presents the expected and observed upper limits in this range. 

The dissertation first presents an overview of the Standard Model and the symmetry breaking mechanism within the theory responsible for the Higgs. After covering the theoretical basis for the Standard Model Higgs, the accelerating apparatus responsible for accelerating and colliding the protons, the LHC, is covered. Then the CMS detector responsible for measuring the paths, momentum, and energy of the particles emerging from every collision is reviewed. Next the machine learning implementation in the L1 trigger that enables the detector to save more of the relevant collisions data is detailed. After, the search for H to $\mu^{+}\mu^{-}$ is presented and then finally the conclusions.
