\chapter{INTRODUCTION} \label{intro}

The Standard Model (SM) of particle physics is an extremely successful theory shown to correctly predict the behavior of the particles and forces which make up the most basic constituents of the universe. In fact, it correctly describes all of the forces known except for gravity. In particular, the SM predicts that the massive particles of the theory acquire their mass by interacting with a scalar particle called the Higgs boson. On July 4, 2012 two collaborations at the Large Hadron Collider (LHC), the A Toroidal LHC Apparatus (ATLAS) and Compact Muon Solenoid (CMS), announced the discovery of a new boson at 125 GeV with properties similar to the Standard Model Higgs \cite{atlasdiscovery,cmsdiscovery2012,cmsdiscovery2013}. This discovery was fueled by the investigation into the Higgs decays to the vector bosons ZZ and ${\rm \gamma\gamma}$. Soon after, evidence for the Higgs coupling to matter was found through the ${\rm \tau^{+}\tau{-}}$ and ${\rm b\bar{b}}$ decays \cite{cmshiggstau,cmshiggsbb,cmshiggsferm,atlashiggsbb}. Whether the newly discovered boson is indeed the expected Standard Model Higgs remains to be determined. Insofar, all of the different decay modes will be investigated to search for deviations from the Standard Model predictions.

This leads to the study of the Higgs decay to $\mu^{+}\mu^{-}$. Although this decay is the smallest branching fraction expected to be detected \cite{smallestbranch1,smallestbranch2}, the dimuon decay offers high efficiency and excellent momentum resolution, which should lead to a narrow peak over the falling background, mostly Drell Yan events. The tiny branching fraction enables greater sensitivity to small deviations from the predicted decay rate and in this respect offers an advantage over other channels where a miniscule deviation could be drowned out. Furthermore, the Higgs coupling to second generation fermions remains to be determined. 

This dissertation presents the search for the Standard Model Higgs Boson decaying to $\mu^{+}\mu^{-}$ using the proton-proton collision data recorded by the CMS experiment in 2016. In order to maximize the data available for the search, the first machine learning in the L1 Trigger system at the LHC was developed and deployed for 2016 data collection. To further maximize the sensitivity of the search, an additional machine learning technique was invented to categorize events based upon the detector resolution and the event kinematics. The search looks for a Higgs boson with a mass between 120 and 130 GeV and presents the expected and observed upper limits in this range as well as the best fit for the rate of production.  

The dissertation first covers the LHC which is responsible for accelerating the colliding the protons. Then the dissertation presents the CMS detector responsible for measuring the paths, momentum, and energy of the emerging particles. Next, the dissertation explains the theory underlying the Standard Model and its predictions of the Higgs particle. After, the machine learning implementation in the L1 trigger that reduced the number of fakes in the data by a factor of three is detailed. And finally, the search for H to $\mu^{+}\mu^{-}$ is presented.
