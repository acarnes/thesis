\chapter{RESULTS FOR THE $H\rightarrow\mu^+\mu^-$ SEARCH} \label{results}

With the categories finalized, the signal and background models pinned down, and the systematic uncertainties set, the results are calculated. The upper limit on the signal strength, the background only p-value, and the best fit of the signal strength are calculated for the 13 TeV data. These results are then combined with the 2012 results from 7 and 8 TeV data. The limits presented use the CLs method.  
\section{Results For 13 TeV Data}
Figures \ref{fig:limitPerCat} to \ref{fig:signalstrength13} present the upper limits, background only p-value, and the best fit signal strength for 13 TeV data. For $m_H$ = 125 GeV, the observed upper limit on the signal strength at 95\% confidence is 2.98, while the expected (median) upper limit if the background only hypothesis is true is 2.44. For $m_H$ = 125 GeV, the observed p-value on the background only hypothesis is 0.62$\sigma$, while the expected p-value on the background only if $H\rightarrow\mu^+\mu^-$ obeys the SM is 0.85$\sigma$. The best fit signal strength at $m_H$ = 125 GeV is $0.7^{+1.3}_{-1.1}$. A weighted average of the S+B fits in each category is shown in Figure \ref{fig:splusbweight}, and the individual fits per category are shown in Figures \ref{fig:fitsbycat1} and \ref{fig:fitsbycat1}. 
\begin{figure}[h!]
    \centering
    \includegraphics[width=0.64\textwidth]{images/results/limit.png}
    \caption[The net upper limit on the signal strength for 13 TeV data alone.]
    {The net upper limit on the signal strength at 95\% confidence is presented for 13 TeV data. The result uses the combined likelihood of all the categories. The red dashed line represents the expected (median) observation if the SM $m_H$ = 125 GeV hypothesis is true. The black dashed line represents the expected observation if the background only hypothesis is true.}
    \label{fig:limit13}
\end{figure}
\begin{figure}[h!]
    \centering
    \includegraphics[width=0.64\textwidth]{images/results/pval.png}
    \caption[The p-value on the background-only hypothesis using 13 TeV data alone.]
    {The p-value is presented for 13 TeV data. The black line is the observed p-value on the background only hypothesis. The blue line represents the expected (median) p-value on the background only if the SM $m_H$ = 125 GeV hypothesis is true. The red line represents the expected p-value at $m_H$ = x GeV if the SM $m_H$ = x GeV hypothesis is true.}
    \label{fig:pval13}
\end{figure}
\begin{figure}[h!]
    \centering
    \includegraphics[width=0.64\textwidth]{images/results/nll_old.png}
    \caption[A plot of the negative log likelihood near the minimum for the signal strength.]
    {The negative log likelihood is plotted near the minimum for the signal strength $\mu$, and the best fit, $\hat{\mu}$, is given.}
    \label{fig:signalstrength13}
\end{figure}
\begin{figure}[h!]
    \centering
    \includegraphics[width=0.64\textwidth]{images/results/CMS-HMM_hmm_combcat_weighted.png}
    \caption[The weighted average of the signal + background fits for 13 TeV data.]
    {The weighted average of the signal + background fits over the categories. The fit from each category is weighted by S/(S+B) to provide an idea of the overall shape while counting those categories with larger signal purity to a greater degree.}
    \label{fig:splusbweight}
\end{figure}
\begin{figure}[h!]
    \centering
    \includegraphics[width=0.48\textwidth]{images/results/CMS-HMM_hmm_hmm_13TeV_cat-00.png}
    \includegraphics[width=0.48\textwidth]{images/results/CMS-HMM_hmm_hmm_13TeV_cat-01.png}
    \includegraphics[width=0.48\textwidth]{images/results/CMS-HMM_hmm_hmm_13TeV_cat-02.png}
    \includegraphics[width=0.48\textwidth]{images/results/CMS-HMM_hmm_hmm_13TeV_cat-03.png}
    \includegraphics[width=0.48\textwidth]{images/results/CMS-HMM_hmm_hmm_13TeV_cat-04.png}
    \includegraphics[width=0.48\textwidth]{images/results/CMS-HMM_hmm_hmm_13TeV_cat-05.png}
    \includegraphics[width=0.48\textwidth]{images/results/CMS-HMM_hmm_hmm_13TeV_cat-06.png}
    \includegraphics[width=0.48\textwidth]{images/results/CMS-HMM_hmm_hmm_13TeV_cat-07.png}
    \caption[Signal + background fits for the individual categories on 13 TeV data.]
    {The final signal + background fits for the 13 TeV data, part one.}
    \label{fig:fitsbycat1}
\end{figure}
\begin{figure}[h!]
    \centering
    \includegraphics[width=0.49\textwidth]{images/results/CMS-HMM_hmm_hmm_13TeV_cat-08.png}
    \includegraphics[width=0.49\textwidth]{images/results/CMS-HMM_hmm_hmm_13TeV_cat-09.png}
    \includegraphics[width=0.49\textwidth]{images/results/CMS-HMM_hmm_hmm_13TeV_cat-10.png}
    \includegraphics[width=0.49\textwidth]{images/results/CMS-HMM_hmm_hmm_13TeV_cat-11.png}
    \includegraphics[width=0.49\textwidth]{images/results/CMS-HMM_hmm_hmm_13TeV_cat-12.png}
    \includegraphics[width=0.49\textwidth]{images/results/CMS-HMM_hmm_hmm_13TeV_cat-13.png}
    \includegraphics[width=0.49\textwidth]{images/results/CMS-HMM_hmm_hmm_13TeV_cat-14.png}
    \caption[More signal + background fits for the individual categories on 13 TeV data.]
    {The final Signal + Background fits for the 13 TeV data, part two.}
    \label{fig:fitsbycat2}
\end{figure}
\clearpage
\section{Results Combining 7, 8, And 13 TeV Data}
Figures \ref{fig:limit7813} to \ref{fig:signalstrength7813} present the upper limits, background only p-value, and the best fit signal strength for the combined 7, 8, and 13 TeV data. For $m_H$ = 125 GeV, the observed upper limit at 95\% confidence is 2.93, while the expected (median) upper limit is 2.16 if the background only hypothesis is true. For $m_H$ = 125 GeV, the observed p-value on the background only hypothesis is 0.91$\sigma$, and the expected p-value if the Higgs is SM is 0.96$\sigma$. The best fit signal strength at $m_H$ = 125 GeV is $0.9^{+1.0}_{-0.9}$. 
\begin{figure}[h!]
    \centering
    \includegraphics[width=0.64\textwidth]{images/results/limit_combined.png}
    \caption[The upper limit on the signal strength combining 7, 8, and 13 TeV data.]
    {The upper limit on the signal strength at 95\% confidence is presented for the combination of 7, 8, and 13 TeV data. The red dashed line represents the expected (median) observation if the SM $m_H$ = 125 GeV hypothesis is true. The black dashed line represents the expected observation if the background only hypothesis is true.}
    \label{fig:limit7813}
\end{figure}
\begin{figure}[h!]
    \centering
    \includegraphics[width=0.64\textwidth]{images/results/pval_combined.png}
    \caption[The p-value on the background only hypothesis combining 7, 8, and 13 TeV data.]
    {The p-value is presented for the combination of 7, 8, and 13 TeV data. The blue line represents the expected (median) p-value on the background only if the SM $m_H$ = 125 GeV hypothesis is true. The red line represents the expected p-value at $m_H$ = x GeV if the SM $m_H$ =x GeV hypothesis is true.}
    \label{fig:pval7813}
\end{figure}
\begin{figure}[h!]
    \centering
    \includegraphics[width=0.64\textwidth]{images/results/nll_old_combined.png}
    \caption[The negative log likelihood for the combined 7, 8, and 13 TeV data.]
    {The negative log likelihood is plotted near the minimum for the signal strength $\mu$, and the best fit, $\hat{\mu}$, is given. The plot uses the combined 7, 8, and 13 TeV data.}
    \label{fig:signalstrength7813}
\end{figure}
