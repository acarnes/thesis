\chapter{THE SEARCH STRATEGY FOR $H\rightarrow\mu^+\mu^-$} \label{strategy}

The goal of the $H\rightarrow\mu^+\mu^-$ search is to discover the decay, and barring that to set an upper limit on the rate of production. The best fit for the rate of production is also reported. In order to do all of this, the analysis needs the data and the expected yields for the signal and background. The expected yields determine the likelihood, the PDF for the test statistic t, and the limits/p-values. The analysis fits Monte Carlo samples along $m_{\mu^+\mu^-}$ for the signal, and fits data along $m_{\mu^+\mu^-}$ for the background. The data at the collected luminosity has far better statistics than the Monte Carlo available for the background, and the larger statistics provide lower uncertainty in the bins of the PDF. 

There is too much data to process to use all of it in the statistical analysis, and that would be wasteful anyways, so basic selections are made to look only at data likely to contain $H\rightarrow\mu^+\mu^-$ signal and unlikely to miss it. With a reasonable amount of data to look at, the Monte Carlo is validated against the data to make sure it agrees with observation. Some corrections like efficiency scale factors, trigger scale factors, and momentum corrections are applied to the Monte Carlo to ensure better agreement with the data. The background Monte Carlo is used to help with the data Monte Carlo validation, and to optimize the sensitivity, but not for the limit setting, discovery p-values, or measurement. 

The analysis attempts to maximize the odds of discovering $H\rightarrow\mu^+\mu^-$ by maximizing the sensitivity. Because the sensitivity increases with the amount of data, improvements to the Level-1 Muon Trigger are made to increase the $\mu^+\mu^-$ data saved for the analysis. Moreover, the sensitivity improves with increased signal and worsens with increased background -- recall that in the limit of large statistics, $Z^2=\sum_i\frac{S_i^2}{B_i}$. As such, the ideal situation is one where the signal is concentrated into a single bin with little or no background. This isn't possible in practice, but regions of feature space may be found where the signal is more concentrated or where there is little background or both. 

The analysis corrects the muon momentum to narrow the $m_{\mu^+\mu^-}$ peak, improving its resolution, and concentrating the signal. To further improve the sensitivity, Boosted Decision Trees (BDTs) \cite{bdt} are trained to separate signal and background. The training produces a discriminating feature with a higher conentration of signal and lower concentration of background at large values. A novel algorithm then optimally and automatically categorizes the events by resolution and BDT score to maximize the expected sensitivity of the analysis. When the S+B fit is along a single dimension like $m_{\mu\mu}$, low sensitivity regions merge with high sensitivity regions of feature space polluting the chance for discovery. The categorization extracts the high sensitivity regions to regain the sensitivity. 

